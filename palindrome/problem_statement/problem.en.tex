\problemname{Base-2 Palindromes}
A positive integer $N$ is a \emph{base-$b$ palindrome} if the base-$b$ representation of $N$ is a palindrome, i.e. reads the same way in either direction.
For instance, $7$ (base $10$) is a palindrome in any base greater than or equal to $8$.
It is also a palindrome in base $2$ ($111$) and $6$ ($11$), but not in $3$ ($21$), $4$ ($13$), $5$ ($12$), or $7$ ($10$).
The first four base $2$ palindromes (written in base $10$) are $1$, $3$, $5$, and $7$.

\section*{Task}
You are supposed to find the $M$-th base-$2$ palindrome and output its base $10$ representation.

\section*{Input}
The input is a single line with a single positive integer $M$ ($M \le 50\,000$) in base $10$.

\section*{Output}
Output a single line with the base $10$ representation of the $M$-th base-$2$ palindrome.
